\begin{tabular}{l|l|l}
    \multicolumn{3}{c}{Тригонометрические функции} \\
    \hline
    
    $\displaystyle sin^2(x) + cos^2(x) = 1$  & 
    $\displaystyle tg^2(x) + 1 = \frac{1}{cos^2(x)}$ &
    $\displaystyle tg(x) = \frac{sin(x)}{cos(x)}$ \\
    
    $\displaystyle tg(x)ctg(x) = 1$ &
    $\displaystyle ctg^2(x) + 1 = \frac{1}{sin^2(x)}$ &
    $\displaystyle ctg(x) = \frac{cos(x)}{sin(x)}$ \\  
    \hline
    
    $\displaystyle sin(x \pm y) = sin(x)cos(y) \pm cos(x)sin(y)$ & 
    $\displaystyle sin(x) = \frac{2tg\left(\frac{x}{2}\right)}{1 + tg^2\left(\frac{x}{2}\right)}$ &
    $\displaystyle sin(2x) = 2sin(x)cos(x)$ \\
    
    $\displaystyle cos(x \pm y) = cos(x)cos(y) \mp sin(x)sin(y)$ & 
    $\displaystyle cos(x) = \frac{1 - tg^2\left(\frac{x}{2}\right)}{1 + tg^2\left(\frac{x}{2}\right)}$ &
    $\displaystyle cos(2x) = cos^2(x) - sin^2(x)$ \\
    
    $\displaystyle tg(x \pm y) = \frac{tg(x) \pm tg(y)}{1 \mp tg(x)tg(y)}$ & 
    $\displaystyle tg(x) = \frac{2tg\left(\frac{x}{2}\right)}{1 - tg^2\left(\frac{x}{2}\right)}$ &
    $\displaystyle tg(2x) = \frac{2tg(x)}{1-tg^2(x)}$ \\
    
    $\displaystyle ctg(x \pm y) = \frac{-1 \pm ctg(x)ctg(y)}{ctg(x) \pm ctg(y)}$ & 
    $\displaystyle ctg(x) = \frac{1 - tg^2\left(\frac{x}{2}\right)}{2tg\left(\frac{x}{2}\right)}$ &
    $\displaystyle ctg(2x) = \frac{ctg^2(x) - 1}{2ctg(x)}$ \\
    \hline
    
    $\displaystyle sin(x) \pm sin(y) = 2sin\left(\frac{x \pm  y}{2}\right)cos\left(\frac{x \pm y}{2}\right)$ &
    $\displaystyle sin\left(\frac{x}{2}\right) = \sqrt{\frac{1 - cos(x)}{2}}$ &
    $\displaystyle sin^2(x) = \frac{1 - cos(2x)}{2}$ \\ 
    
    $\displaystyle cos(x) + cos(y) = 2cos\left(\frac{x +  y}{2}\right)cos\left(\frac{x - y}{2}\right)$ &
    $\displaystyle cos\left(\frac{x}{2}\right) = \sqrt{\frac{1 + cos(x)}{2}}$ &
    $\displaystyle cos^2(x) = \frac{1 + cos(2x)}{2}$ \\  
    
    $\displaystyle sin(x) \pm sin(y) = 2sin\left(\frac{x \pm  y}{2}\right)cos\left(\frac{x \pm y}{2}\right)$ &
    $\displaystyle tg\left(\frac{x}{2}\right) = \sqrt{\frac{1 - cos(x)}{1 + cos(x)}}$ &
    $\displaystyle (sin(x) + cos(x))^2 = 1 + sin(2x)$ \\  
    \hline
    
    $\displaystyle sin(x)sin(y) = \frac{1}{2}(cos(x - y) - cos(x + y))$ &
    \multicolumn{2}{l}{$\displaystyle sin(x) = x - \frac{x^3}{3!} + \frac{x^5}{5!} - \frac{x^7}{7!} + \ldots + (-1)^n\frac{x^{2n + 1}}{(2n + 1)!} + o(x^{2n+2})$}  \\  
    
    $\displaystyle cos(x)cos(y) = \frac{1}{2}(cos(x - y) + cos(x + y))$ &
    \multicolumn{2}{l}{$\displaystyle cos(x) = 1 - \frac{x^2}{2!} + \frac{x^4}{4!} - \frac{x^6}{6!} + \ldots + (-1)^n \frac{x^{2n}}{(2n)!} + o(x^{2n+1})$}  \\  
    
    $\displaystyle sin(x)cos(y) = \frac{1}{2}(sin(x - y) + sin(x + y))$ & 
    \multicolumn{2}{l}{$\displaystyle e^x = 1 + x + \frac{x^2}{2!} + \frac{x^3}{3!} + \ldots + \frac{x^n}{n!} + o(x^n)$}  \\ \cline{1-1}  
    
    \multicolumn{3}{r}{$\displaystyle (1 + x)^m = 1 + mx +\frac{m(m-1)}{2!}x^2 + \ldots + \frac{m(m-1)\ldots[m-(n-1)]}{n!}x^n + o(x^n)$}  \\   
    \cline{1-1}
    
    $\displaystyle sin(3x) = 3sin(x) - 4sin^3(x)$ &
    \multicolumn{2}{l}{$\displaystyle ln(1+x) = x - \frac{x^2}{2} + \frac{x^3}{3} + \frac{x^4}{4} + \ldots + (-1)^{n-1}\frac{x^n}{n} + o(x^n)$}  \\  
    
    $\displaystyle cos(3x) = 4cos^3(x) - 3cos(x)$ &
    \multicolumn{2}{c}{} \\
    
    $\displaystyle tg(3x) = \frac{3tg(x) - tg^3(x)}{1 - 3tg^2(x)}$ &
    \multicolumn{2}{c}{} \\
    \hline
    
    %    $\displaystyle $ &
    %    $\displaystyle $ &
    %    $\displaystyle $ \\     
\end{tabular}